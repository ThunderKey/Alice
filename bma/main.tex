%%%%%%%%%%%%%%%%%%%%%%%%%%%%%%%%%%%%%%%%%
% Large Colored Title Article
% LaTeX Template
% Version 1.0 (15/8/12)
%
% This template has been downloaded from:
% http://www.LaTeXTemplates.com
%
% Original author:
% Frits Wenneker (http://www.howtotex.com)
%
% License:
% CC BY-NC-SA 3.0 (http://creativecommons.org/licenses/by-nc-sa/3.0/)
%
%%%%%%%%%%%%%%%%%%%%%%%%%%%%%%%%%%%%%%%%%

\input{parts/layout}

\begin{document}

\newcommand{\printhref}[2]{
  \href{#1}{#2 (#1)}
}

\begin{titlepage}
  \thispagestyle{empty}
  \maketitle % Print the title
\end{titlepage}

%\thispagestyle{plain}

\tableofcontents
%----------------------------------------------------------------------------------------
%	ABSTRACT
%----------------------------------------------------------------------------------------

\chapter{Abstract}
% The first character should be within \initial{}
\initial{H}\textbf{ere is some sample text to show the initial in the introductory paragraph of this template article. The color and line height of the initial can be modified in the preamble of this document.}

%----------------------------------------------------------------------------------------
%	ARTICLE CONTENTS
%----------------------------------------------------------------------------------------

\chapter{Introduction}


%------------------------------------------------

\section{Section 1}


\begin{itemize}
  \item First item in a list 
  \item Second item in a list 
  \item Third item in a list
\end{itemize}


%------------------------------------------------

\section{Section 2}


%------------------------------------------------
\chapter{Artificial Intelligence}
\section{What is AI?}

This question has about as many answers as there are people studying artificial intelligence. Without getting too philosophical one could say that intelligence is defined by the efficiency of patterns used to solve problems in a complex environment and the ability to learn or generate new patterns by observing or testing.

Man-made programs to solve such problems have been around for decades and as algorithm efficiency and computational power increases they get ever more powerful.

An AI usually is directed towards one goal. This could mean solving a mathematical challenge most efficiently, finding patterns in huge amounts of data to categorize new sets of data or even to find smarter ways of learning.


\section{History of the Field}

\section{Technological Singularity}

\newpage
\section{Real World Examples}

\subsection{Amazon Recommendations}
Everyone knows the ``People who bought X also bought Y''-kind of recommendations online shops provide you with. Sometimes they are spot on and other times they are as far off as they can be. The idea of course is to find patterns in customer interest which can be used to offer people things they are likely to buy thereby increasing profit.

So how do they work? It is all about finding similarities in data. In the case of products you have attributes like cost, size, style, manufacturer, etc. that let classify the product. Yet taken for themselves they are close to useless because two products that are similar in size and cost might not have anything in common at all. It's even worse. Imagine two lamps that have almost the same price, the same size and the same style. What's the probability that someone who bought the one lamp would also want to buy the other? Close to zero. Now why is that? For one thing it's unlikely that someone needs two lamps. It's even less likely that they would need two almost identical lamps. So apart from flat stats about a product one needs to take into account the nature of the product to make good recommendations. An example of such would be matching light bulbs for the lamp. 
This tells us two things. First we need to categorize a product. Not only in it's use (e.g. lamp, chair, table, clock) but also in what frequency demand for it arises. So for example one would like to buy a new shirt every month whereas a new computer is only required every few years. The second thing is that there are certain items which depend on other items to function. A coffee machine needs filters, a toy car needs batteries and a lamp needs light bulbs. Hence to make a smart suggestion one would link those items one way so that buying a laptop would yield to recommendation of a mouse but buying cat food would not recommend acquiring a new feline companion.

This probably all sounds perfectly reasonable. Every halfway decent online shop should be capable handling categories and dependencies between products. Since the assignment of such seems more or less obvious even to an intern why not just hire some of those to do it? It doesn't require an expensive AI to figure out that it would be a good idea to recommend "Harry Potter" part two to someone who bought part one. Yet if those where to only clues to go on the list of suggestions would be pretty short for most products. To make matters worse: Since those types of relations are so obvious, it's likely the customer thought of them already so they don't yield as much additional sales as "smart" recommendations would. This is where the AI algorithms come in.

Imagine what a resale company could do if they understood their customer's taste. If they knew what style of clothing she likes, whether she lives near a beach and thus is more likely to buy a new bathing suit in spring or when her mothers birthday was so they could recommend gifts for older women at that time of year. Admittedly these examples might are little extreme yet the efforts of the so called data miners definitely go in that direction. These three examples have been picked for a reason because they each depend on different types of data.

The style for example can be derived from the clothes the customer has bought in the past. From those the retailer could categorize his customers into groups that appear to like the same products and recommend things person A from this group bought to person B. Of course there is more to it than that. For example one needs to filter out noise which may be a piece of clothing a customer bought for someone else and thus does not match the pattern. Now an especially smart AI could even make a connection between such purchases and the friend or family member they where meant to. This data then could be used for a variety of further predictions about the customers buying behavior.

The fact whether someone lives near a beach (or swimming pool for that matter) can be easily determined by matching the IP address from which requests to the store website were made to the corresponding location. There is a lot more of this type of information that one can easily read from the web request like browser type, mobile phone model, referring website and by matters of recurring visits from different locations even estimates about the customer's place of work or their boyfriend's address. Now this type of information is only as useful as one is able to make sense of it. This is the true strength of an AI because it can find patterns in data that no one has been looking for so far.

\subsection{Web Search Algorithms}

In web search it is even more important than with large selections of products to have an automated way to make precise estimations about the value of something to someone. Furthermore the search provider not only wants to establish a prediction about how likely a person is to be looking for a certain web page but also about how likely he or she is to be looking for it right now. A lot of things factor into this like previously visited websites, behavior in choosing links and probably over a thousand other data points that might have been collected about a person. Yet even if there is no previous information about a search engine user available they can still make predictions based on the location, language and outside factors. An example of this might be the following scenario:

A user sits down in a Boston coffee shop using a browser, the signature of which has been modified to prevent it from being tracked back to that certain individual, to access her search provider of choice. She types in the word "dolphins". Now the first assumptions is that she is looking for the marine mammals. Yet the night before the football team by the same name might have played which would vastly improve the relevance of last nights game results. This increase of relevance can be justified by the fact that the user has set her language preference to American English which makes it far more likely that she is interested in the score than someone with a preference for French would be. But this isn't everything that is relevant to the search. Assuming that a band called "The Dolphins" is playing this week in Boston would make the band's website and a ticket sales site a lot more relevant to this search then it would be for someone from San Francisco.

Again this example shows how a shift in relevance might be obvious if pointed out but is extremely hard to predict for a human. Artificial intelligence can look at changes in user behavior (like increased requests to "The Dolphins" band website from Boston) and smartly adjust the ranking of certain websites. Of course this is an oversimplification of the problem but it shows nicely the basic idea of using AI to find patterns in changing data.

\subsection{DARPA Challenges}

The Defense Advanced Research Projects Agency has funded prize competitions that reward advances in research fields relevant to the Department of Defense. The initial challenge was to build autonomous ground vehicles or "self-driving cars". It was called the DARPA Grand Challenge and was first held in 2004. The goal was to build an autonomous vehicle that could drive the 240 km route through the desert fastest. None of the robot vehicles that entered the contest was able to complete the challenge. The one that performed best only finished five percent of the route before getting hung up on a rock. Therefore the event was repeated in 2005.

The second time around five vehicles were able to finish the race. With the team from Stanford University finishing first. 

In 2007 the DARPA Urban Challenge has been held requiring the vehicles to finish a 96 km course in an urban area while obeying traffic regulations and interacting with human drivers.

Currently Google is developing a driverless car which has already driven 300'000 miles without an accident. The project is led by Sebastian Thrun who also was head of the Stanford team that won the 2005 DARPA Grand Challenge.

\subsection{Deep Blue}

In 1997 the IBM chess computer Deep Blue famously defeated the world champion Garry Kasparov.

\subsection{IBM Watson}

\section{Traditional AI vs Neural Networks}

\newpage
\section{Some AI Algorithms in Detail}
\subsection{Graph Search}
\subsection{Beyas Networks}

\[P(A|B)=\frac{P(B|A) * P(A)}{P(B)}\]

\subsection{Markov Decision Processes}
\subsection{Natural Language Learning}
\subsection{Particle Filters}
\subsection{Heuristics}
\subsection{Game Theory}

\chapter{Project}
What do we need to create a 2 dimensional jump and run game? First of all we need a physics engine to recreate a realistic environment. Then we need a game engine to build our game on. After that we need the game itself and an environment creator to randomly generate the worlds.\bigskip\\
Now we need an AI. But how do we connect the AI with the game? How can the AI controll the player? How can the AI see where the next block is? That's where the Application Programming Interface (API) steps in. With this connection the AI can easily read the environment and move the player. Lastly we ``just'' need the AI algorithms.

\begin{figure}[H]
  \includegraphics[width=0.45\textwidth]{images/game.png}
  \caption{The final game at the start point.}
\end{figure}
\newpage

\section{Physics Engine}
A physics engine is able to recreate an entire physical system like our world. They are mostly common in animated films, scientific experiments and video games. The gravity is the biggest part to cover. There are less and more accurate engines. In some the falling objects get faster or the objects can break when they fall on a solid ground.

\subsection{Gravity Example}
A block gets thrown horizontally with the force F. This block has now the force from the throw (horizontally) and from the gravity (vertically).\bigskip\\

\begin{figure}[H]
  \includegraphics[width=0.45\textwidth]{images/gravity_example.png}
  \caption[Gravity example from \url{http://www.sparknotes.com/testprep/books/sat2/physics/chapter7section1.rhtml}]{Example for gravity.}
\end{figure}

\subsection{Collision Example}
A block gets thrown horizontally with the force F and collides with another, steady one with the same size and weight. After the collision the first block will fly back with half the force, while the other one gets pushed away with the other half of the force.

\begin{figure}[H]
  \includegraphics[width=0.45\textwidth]{images/collision_example.png}
  \caption[Collision example from \url{http://physicslearning2.colorado.edu/pira/resources/physics-testlecture-drawings}]{Example for a collision.}
\end{figure}

\subsection{Types of Physics Engines}
There are two main types of physics engines. The high-precision and the real-time engines:
\begin{description}
  \item[high-precision] This type of engine is used for exact copies of a physical system. They are mostly common for animations that must be very precise like scientific studies or animated films. It does not matter in this animations how many resources where used while rendering because you can let the machine run for many hours rendering the world and than analyze the result at the end.\smallskip\\
    To wrap it up: Those engines need much processing powers but are very precise. \ul{Quality before quantity}.
  \item[real-time] This type of engine is used for interactive animations like video games. The artificial physics must be calculated immediately. Unlike the high-precision engines this type of engine does not need much processing power, for it is minimized.\smallskip\\
    To wrap it up: The real-time engines do not need much processing power but are not very accurate. \ul{Quantity before quality}.
\end{description}

\subsection{What do we need?}
\begin{description}
  \item[collision] We need to be able to check for a collision between the player and a block above the ground.
  \item[gravity] We need to be able to use gravity for the player to jump and fall down. 
  \item[real-time] We are creating a game, so we need in the first place a fast engine.
\end{description}
\subsection{Setting Boundaries}
But let us stop here. Even though it is a really interesting topic, it is too extensive to go into the details. If you would like to know more, there are plenty of comprehensive studies about physics engines.
\newpage

\section{Game Engine}
We want the physics engine to create a game. But how can one do this? All by himself? No, we need to find a link between the game and the physics engine. This is where the game engine has a turn. It offers the opportunity never to touch the physics engine and to make the step from a world with just moving objects to a world, controlled by a human.
\subsection{What does a Game Engine?}
\begin{itemize}
  \item It unburdens the interaction between the human's controls like a keystroke or a mouse click and the animations that can be seen on the display.
  \item It eases the animations of the objects. For example when the player walks then should the character animate this movement.
  \item It implements the element of sound. With this feature you can easily add background music and/or sound effects.
  \item They are mostly specialized on 2D or 3D worlds. 3D worlds are way more complicated but they are more realistic.
  \item They are either built for a specific platform or they can compile the same source code for multiple platforms. For example if it supports multiple platforms you can create a game and run it on Windows, Linux, Xbox and PlayStation.
  \item Some engines even include a AI system to help building one. They often helps with the data logistics and sensor access.
\end{itemize}

\subsection{Examples of Game Engines}
\begin{description}
  \item[CryENGINE] A engine created by Crytek. It runs on PS3, Xbox 360 and PC. It has an advanced AI system with sensors like sight and hearing. The games Crysis, FarCry and Warface are the most famous of many.
  \item[Unreal Engine] A engine created by Epic Games. It runs on PC, PS3, Xbox 360, Wii U, PSVITA, Android, iOS and Flash. The Unreal AI is also advanced and includes a special navigation mesh system to optimize the performance and memory usage. Dishonored, Gears of War, Batman: Arkham City and Mass Effect are the best examples for this engine.
  \item[Anvil] Until 2006 it was known as Scimitar and is created by Ubisoft. It runs on PC, PS3, Xbox 360, Wii U and PSVITA. Prince of Persia, Assassin's Creed and Tom Clancy's Rainbow 6: Patriots are the most known examples for this engine.
  \item[IW Engine] A engine created by Infinity Ward. It runs on PC, PS3, Xbox 360, Wii and Wii U. The complete Call of Duty series expect the first one is based on this engine.
  \item[Blender] This is the most popular open source game engine. 
\end{description}

\subsection{What do we need?}
We don't need a complex game engine for we are creating a simple, 2 dimensional game. It should just cover the basic necessities like:
\begin{itemize}
  \item interaction between user interface and animations
  \item animations for the objects
  \item sound implementations
  \item built for PC only should be sufficient
  \item 2 dimensional
\end{itemize}
\newpage

\section{The Game}
For it would take too much time to create an own physics and game engine, we decided to use the open source engine Crafty. It is based on the programming language JavaScript allowing the game to run in a modern web browser.

\subsection{Goal}
  We want to create a game with this specifications:
  \begin{itemize}
    \item 2 dimensional
    \item jump and run
    \item blocks to jump on
    \item gaps in the floor to fall down
    \item randomly generated
    \item generator ensures that the game is solvable
    \item finish line at the end of the parcour
    \item die from falling down
    \item win from reaching the finish line
    \item after winning or dying: restart the game
  \end{itemize}

\subsection{Basic Composition}
We used an event-driven development architecture. That means that the engine triggers events and we define what should happen.\bigskip\\
Here is the main structure of the game in a theoretically programming language:
\begin{verbatimtab}
When the player hits a solid object:
  The player stops moving

When the player hits a deadly area:
  The player dies
  Restart the game

When the player hits the finish line:
  The player wins
  Restart the game
\end{verbatimtab}


\subsection{Animation Sprites}
To bring life to our player we decided to use a sprite. A sprite is basically a collection of images with the same character but in different positions. You just have to decide which of them belongs to which movement.\bigskip\\

\begin{figure}[H]
  \includegraphics{images/player.png}
  \caption{The sprite to animate our player.}
\end{figure}

\begin{description}
  \item[standing right] If the player stands and looks to the right the game should use the image from \ul{row 1 the first}
  \item[standing left] If the player stands and looks to the left the game should use the image from \ul{row 2 the first}
  \item[jumping right] If the player jumps to the right the game should use the image from \ul{row 1 the last}
  \item[jumping left] If the player jumps to the left the game should use the image from \ul{row 2 the last}
  \item[walking right] If the player walks to the right the game should switch through all images from \ul{the third row}
  \item[walking left] If the player walks to the left the game should switch through all images from \ul{the forth row}
\end{description}

\newpage
\subsection{Environment Creator}
Our environment consists of blocks and a floor with gaps. We now need a creator that generates random environments. To make it consistent we chose that all blocks have the same size. They are in one of this three heights:
\begin{enumerate}
  \item The height of the player. Easy to jump on.
  \item Twice the height of the player. The maximal height to jump on.
  \item Thrice the height of the player. Just reachable from a block from the first or second height.
\end{enumerate}
\begin{figure}[H]
  \includegraphics[width=0.45\textwidth]{images/block_heights.png}
  \caption{The three possible block heights.}
\end{figure}
The gaps in the floor exist in three different sizes:
\begin{description}
  \item[one block] These gaps are very easy to jump over.
  \item[two blocks] These are a little harder, but should pose no problem.
  \item[three blocks] These are the most difficult ones. To jump over them you have to stand on the very last corner of the floor or on a block.
\end{description}
\begin{figure}[H]
  \includegraphics[width=0.145\textwidth]{images/gap_size1.png}
  \includegraphics[width=0.145\textwidth]{images/gap_size2.png}
  \includegraphics[width=0.145\textwidth]{images/gap_size3.png}
  \caption{The three possible sizes of the gaps.}
\end{figure}

\subsubsection{Prevent Unsolvable Environments}
Basically we could just say ``create some blocks and some gaps in the ground'' but it would most likely get unsolvable. To prevent this we have chosen this insurances to prevent that:
\begin{itemize}
  \item The blocks from the first or second height are no problem for the player to jump on.
  \item The blocks on the third height may not exist if they don't have an accessible block from the first or second height. This is to make sure that the player can jump on every block.
  \item The gaps may not be bigger than the player could jump. 
\end{itemize}

\subsubsection{Composition}
This is the basic composition of the creator in a theoretical programming language:
\begin{verbatimtab}
Define the start point
Until the whole width of the field is used:
  Set a random end point
  Create a ground from the start to end point
  Get a random gap width (1, 2 or 3)
  Set the start point: end point + gap width

15 times:
  Until the x-coordinate is unique:
    Generate random x-coordinate

  Get a random height of 1 or 2
  Generate the block from x and the height

  With a probability of 3:
    Generate a next block with the height 3
\end{verbatimtab}


\newpage
\section{Application Programming Interface (API)}
As mentioned before we need to create an interface for the AI to control the player and read the environment. To gain full control we need actors and sensors.\medskip\\
For the human the actors are the keys and his fingers with which he can handle the player. The sensors are the monitor and his eyes. With this he can analyze what is happening.\medskip\\
Now we have to adapt those actors and sensors to our API.
\subsection{Actors}
The actors are responsible for the actions of the player. We need those available actions:
\begin{description}
  \item[jump] The AI must be able to jump to the left and the right.
  \item[walk] The AI must be able to walk to the left and the right.
\end{description}

\subsection{Sensors}
The sensors are responsible for the measuring of the environment. We need those available sensors:
\begin{description}
  \item[won?] Did the player win?
  \item[died?] Did the player die?
  \item[my position] Where is the position of the player?
  \item[nearest block] Where is the nearest block?
  \item[all blocks] Where are all blocks? This is needed to plan a few steps further
  \item[nearest gap] Where is the nearest gap?
  \item[all gaps] Where are all gaps?
  \item[finish line] Where is the finish line? To check if the AI is going in to the right direction.
\end{description}
\subsection{Implementation}
Those actors and sensors sound all very logical but how could we implement those?\medskip\\
For the sensors we just have to access the created elements. Of course they need to be well formed and easily accessible. For the nearest gap or block we just have to iterate through all and check which of those the closest one is.\medskip\\
For the actors it is more complicated. We have to include multiple callbacks and helper variables to ease the access. For example we need a callback that will be triggered when the player lands on the ground or when he have reached the desired distance.
\newpage
\section{Our Artificial Intelligence (AI)}

Without going too much into detail about the code, the idea behind our AI is this. Given that there are five possible actions (jump, jump right, jump left, walk left and walk right) which the player can perform, each with the parameter of length, the AI has a lot of options to choose from. One way to decide what action should be performed is to try different ones and look at the result. For example being closer to the goal line is a positive result while falling into a hole is a negative result. By trial and error the agent can learn how far and high it can jump. After it learned this it knows how far from a hole it needs to jump. 

Yet there are also blocks to be considered so we taught the AI a simple rule: "The higher you are, the better". Applying this rule the AI jumps on every block possible thereby eliminating the need for calculating complicated evasion strategies for the blocks in its jump line.
This shows that there are basically two phases in the life cycle for this AI. First it learns about its abilities and the possible distances. Second it performs actions which are chosen depending on these parameters and plays the game.

Admittedly we used some heuristics here that probably weren't necessary. For example the AI knows in what direction the goal is and thus can value moves to the right higher than moves to the left without every having seen the goal. One could remove this prior knowledge and let the AI find the goal on it's own. One could even go further and remove any heuristics apart from the goal factor and let the AI perform random moves until it finds a goal. Based on the data it collected during that time it could calculate all the rules we taught it. Yet we decided against implementing it this way because it would take a lot of time for it to find the goal with more or less random moves and it wouldn't have any real advantage to solving the problem since the AI doesn't need to be this adaptable. It's an important lesson for AI and computer science in general never to overcomplicate things if an increase in the problems complexity isn't foreseeable.


%----------------------------------------------------------------------------------------

\chapter{Sum up}


\chapter{Sources and Glossary}

\section{Sources}
\begin{description}
  \item[Physics Engines] \hfill \\
    \begin{itemize}
      \item \printhref{http://en.wikipedia.org/wiki/Physics\_engine}{Physics engine - Wikipedia}
    \end{itemize}
  \item[Game Engines] \hfill \\
    \begin{itemize}
      \item \printhref{http://en.wikipedia.org/wiki/Game\_engine}{Game engine - Wikipedia}
      \item \printhref{http://www.crytek.com/cryengine}{CryENGINE $|$ Crytek}
      \item \printhref{http://www.unrealengine.com}{Game Engine Technology by Unreal}
      \item \printhref{http://en.wikipedia.org/wiki/Anvil\_(game\_engine)}{Anvil (game engine) - Wikipedia}
      \item \printhref{http://en.wikipedia.org/wiki/IW\_engine)}{IW engine - Wikipedia}
    \end{itemize}
  \item[The Game] \hfill \\
    \begin{itemize}
      \item \printhref{http://craftyjs.com/}{Crafty - Javascript Game Engine}
    \end{itemize}
\end{description}

\printglossary%[type=main,title={Glossary},toctitle={Glossary}]
\addcontentsline{toc}{section}{Glossary}

\section{Image Sources}
\listoffigures

\end{document}
