%%%%%%%%%%%%%%%%%%%%%%%%%%%%%%%%%%%%%%%%%
% Large Colored Title Article
% LaTeX Template
% Version 1.0 (15/8/12)
%
% This template has been downloaded from:
% http://www.LaTeXTemplates.com
%
% Original author:
% Frits Wenneker (http://www.howtotex.com)
%
% License:
% CC BY-NC-SA 3.0 (http://creativecommons.org/licenses/by-nc-sa/3.0/)
%
%%%%%%%%%%%%%%%%%%%%%%%%%%%%%%%%%%%%%%%%%


%----------------------------------------------------------------------------------------
%	PACKAGES AND OTHER DOCUMENT CONFIGURATIONS
%----------------------------------------------------------------------------------------

\documentclass[DIV=calc, paper=a4, fontsize=11pt, twocolumn]{scrartcl}	 % A4 paper and 11pt font size

\usepackage{lipsum} % Used for inserting dummy 'Lorem ipsum' text into the template
\usepackage[english]{babel} % English language/hyphenation
\usepackage[protrusion=true,expansion=true]{microtype} % Better typography
\usepackage{amsmath,amsfonts,amsthm} % Math packages
\usepackage[svgnames]{xcolor} % Enabling colors by their 'svgnames'
\usepackage[hang, small,labelfont=bf,up,textfont=it,up]{caption} % Custom captions under/above floats in tables or figures
\usepackage{booktabs} % Horizontal rules in tables
\usepackage{fix-cm}	 % Custom font sizes - used for the initial letter in the document
\usepackage[utf8x]{inputenc}

\usepackage{sectsty} % Enables custom section titles
\allsectionsfont{\usefont{OT1}{phv}{b}{n}} % Change the font of all section commands

\usepackage{soul} % for highlighting, underlining etc.
\usepackage{hyperref} % for hyperlinks
\usepackage{graphicx} % for graphics
\usepackage{moreverb} % code snippets
\usepackage{float}

\usepackage{lettrine} % Package to accentuate the first letter of the text
\newcommand{\initial}[1]{ % Defines the command and style for the first letter
  \lettrine[lines=3,lhang=0.3,nindent=0em]{
    \color{DarkGoldenrod}{
      \textsf{#1}
    }
  }{}
}

%----------------------------------------------------------------------------------------
%	LAYOUT SECTION
%----------------------------------------------------------------------------------------

\usepackage{lastpage} % Used to determine the number of pages in the document (for "Page X of Total")
\usepackage{fancyhdr} % Needed to define custom headers/footers
\fancypagestyle{plain}{% Overwriting the plain, so the table of contents will use it too
  \hypersetup{
    colorlinks=true
  }

  % Headers
  \fancyheadoffset{0.5 cm}
  \lhead{Timetable}
  \chead{}
  \rhead{\today}

  % Footers
  \fancyfootoffset{0.5 cm}
  \lfoot{Michael Gerber, Nicolas Ganz}
  \cfoot{}
  \rfoot{\footnotesize Page \thepage\ of \pageref{LastPage}} % "Page 1 of 2"

  \setlength{\topmargin}{-30pt}
  \addtolength{\textheight}{-40pt}
  \setlength{\columnsep}{1.5cm}
  \setlength{\columnseprule}{0.4pt}

  \renewcommand{\headrulewidth}{0.4pt} % Thin header rule
  \renewcommand{\footrulewidth}{0.4pt} % Thin footer rule
}

\pagestyle{plain}

%----------------------------------------------------------------------------------------
%	GLOSSARY SECTION
%----------------------------------------------------------------------------------------

\usepackage[toc]{glossaries} % for the glossary
\makeglossaries

\newglossaryentry{ai}{
  name={Artificial Intelligence (AI)},
  description={is a program that learns and gains experience in doing an activity like a game.}
}


%----------------------------------------------------------------------------------------
%	TITLE SECTION
%----------------------------------------------------------------------------------------

\usepackage{titling} % Allows custom title configuration

\newcommand{\HorRule}{\color{DarkGoldenrod} \rule{\linewidth}{1pt}} % Defines the gold horizontal rule around the title

\pretitle{
  \vspace{-30pt} \begin{flushleft} \HorRule \fontsize{50}{50} \usefont{OT1}{phv}{b}{n} \color{DarkRed} \selectfont
} % Horizontal rule before the title


\posttitle{\par\end{flushleft}\vskip 0.5em} % Whitespace under the title

\preauthor{
  \begin{flushleft}\large \lineskip 0.5em \usefont{OT1}{phv}{b}{sl} \color{DarkRed}
} % Author font configuration

\postauthor{
  \footnotesize \usefont{OT1}{phv}{m}{sl} \color{Black}
    % Configuration for the institution name
    BMS Zürich % Your institution
  \par\end{flushleft}\HorRule
} % Horizontal rule after the title


\begin{document}

\newcommand{\printhref}[2]{
  \href{#1}{#2 (#1)}
}


\title{Time Schedule} % Your article title
\author{Michael Gerber, Nicolas Ganz } % Your name
\date{\today}
\maketitle % Print the title

\begin{description}
  \item[04.08.2012] Idea for the project.
  \item[04.08 - 10.10.2012] Learning about AI.
  \item[01.09 - 09.09.2012] First steps in \LaTeX:
    \begin{itemize}
      \item Research about \LaTeX
      \item Trying by creating simple example documents.
      \item Creating our layouts for the documents. 
    \end{itemize}
  \item[09.09 - 10.09.2012] Searching a physics / game engine.
  \item[11.09 - 02.10.2012] Building a basic game:
    \begin{itemize}
      \item Ability to walk.
      \item Ability to jump.
      \item Animation of the player.
      \item A static world with one floor and some blocks.
    \end{itemize}
  \item[27.09 - 03.10.2012] Randomizing the world:
    \begin{itemize}
      \item Random blocks.
      \item Random gaps.
    \end{itemize}
    \newpage
  \item[02.10 - 04.10.2012] Creating the API:
    \begin{itemize}
      \item Actors:
        \begin{itemize}
          \item jump
          \item walk
        \end{itemize}
      \item Sensors:
        \begin{itemize}
          \item get all blocks
          \item get the nearest block
          \item get all gaps
          \item get the nearest gap
          \item get the players position
          \item get the distance to the finish line
        \end{itemize}
    \end{itemize}
  \item[10.10 - 20.10.2012] Creating a simple AI. We've decided that we firstly just want to create an AI that solves the game without the blocks. Then we should see what it really means to create an AI.
  \item[20.10 - 15.11.2012] Creating a more complex AI.\\Now it's time to include the blocks.
  \item[15.11 - 20.11.2012] Finishing the documents.
\end{description}
\end{document}
